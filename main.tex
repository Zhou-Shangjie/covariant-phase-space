\documentclass[10pt]{article}
\usepackage{zsj}

\usepackage{silence}
\WarningFilter{latex}{Marginpar on page}

\allowdisplaybreaks

\hbadness=99999

\newcommand{\me}{\mathrm{e}}
%\newcommand{\rr}{\mathbb{R}}
\newcommand{\ii}{\mathrm{i}}
%\newcommand{\md}{\mathrm{d}}
%\DeclareMathOperator{\im}{im}


\usepackage{marginfix}
\usepackage[export]{adjustbox}

\crefname{claim}{claim}{claims}


\begin{document}
\title{Covariant Phase Space}
\subheader{Notes}
\author{Shangjie Zhou\orcidlink{0000-0001-9576-5011}}
\affiliation{Department of Physics and Astronomy, Purdue University, West Lafayette, IN 47907, USA}
\emailAdd{ZhouShangjie@purdue.edu}
\abstract{\textit{Last updated on: \today}\\Latest version: \url{https://github.com/Zhou-Shangjie/covariant-phase-space/releases}\\Source files: \url{https://github.com/Zhou-Shangjie/covariant-phase-space}\\Personal Website: \url{https://zhou-shangjie.com}}
\maketitle
\phantomsection\addcontentsline{toc}{section}{\protect\numberline{}Introduction}
\section*{Introduction}
This note follows \cite{Harlow:2019yfa}

\section{Hamiltonian Formalism}
The Hamilton equations are
\begin{subequations}
    \begin{align}
        \dot{q}^a&=\pdv{H}{p^a}\\
        \dot{p}^a&=-\pdv{H}{q^a}
    \end{align}
\end{subequations}
The usual Hamiltonian equation splits time coordinate $t$ from other coordinates which make it hard to maintain covariance.

We view the phase space as an manifold $\mathcal{P}$ equipped with an non-degenrate\sidenote{That $\Omega$ is degenrate means: there is a vector $X\in V(\mathcal{P})$ makes that $\forall Y\in V(\mathcal{P})$, $\Omega(X,Y)=0$.} two-form $\Omega:V(\mathcal{P})\times V(\mathcal{P})\to\mathbb{R}(\mathcal{P})$ called the \textit{symplectic form}.
From $\Omega$, we can induce a map $\Omega_{\text{ind}}:V(\mathcal{P})\to V^\ast(\mathcal{P})$ and for $X,Y\in V(\mathcal{P})$ we define 
\begin{equation}
    \Omega_{\text{ind}}(X)(Y)\equiv\Omega(X,Y)
\end{equation}
and notice that $\Omega_{\text{ind}}(X)\in V^\ast$.
We can also define $\Omega^{-1}:V^\ast(\mathcal{P})\times V^\ast(\mathcal{P})\to \mathbb{R}(\mathcal{P})$ via\sidenote{Notice the abuse of notation: $\Omega^{-1}$ here is actually not the inverse of $\Omega$.}
\begin{equation}
    \Omega^{-1}(\omega,\sigma)\equiv\omega(\Omega_{\text{ind}}^{-1}(\sigma))
\end{equation}
where $\omega,\sigma\in V^\ast(\mathcal{P})$.
Given a function $H\in\mathcal{P}$, we can define $X_H\in V(\mathcal{P})$ as 
\begin{equation}
    X_H(f)\equiv\Omega^{-1}(\delta f,\delta H)
\end{equation}
where $f\in\mathcal{F}(\mathcal{P})$.

\begin{definition}[Poisson bracket]
    On manifold $\mathcal{P}$, the \textit{Poisson bracket} of two functions  $f,g\in\mathcal{F}(\mathcal{P})$ is defined as  
    \begin{equation}
        \pb{f}{g}\equiv\Omega^{-1}(\delta f,\delta g)=\Omega(X_g,X_f).
    \end{equation}
\end{definition}
Now we require that in a dynamic system, the time evolution of a function $f\in\mathcal{F}(\mathcal{P})$ to satisfy 
\begin{equation}
    \dot{f}=X_H(f)=\pb{f}{H}.
\end{equation}
The usual Hamiltonian mechanics can be restored if we take 
\begin{equation}
    \Omega=\sum_a \delta p^a\wedge \delta q^a
\end{equation}

\section{Construction of Covariant Phase Space}
We consider the following action 
\begin{equation}
    S=\int_M L+\int_{\partial M}\ell
\end{equation}

If we decompose $\partial M=\Gamma\cup\Sigma_+\cup\sigma_-$, the variation of action around classical solutions should be 
\begin{equation}
    \delta S=\int_{\Sigma_+}\Psi-\int_{\sigma_-}\Psi\label{var_S_1}
\end{equation}

By integration by parts, any local Lagrangian can be written in the following form 
\begin{equation}
    \delta L=E_a\delta \phi^a+\dd{\Theta}
\end{equation}

Therefore, we can write 
\begin{equation}
    \delta S=\int_M E_a\delta\phi^a+\int_{\partial M}\left(\delta\ell+\Theta\right).\label{var_S_2}
\end{equation}
For a variation around classical solution $\phi_c$, since \cref{var_S_2} should be in the same form as \cref{var_S_1} and $\delta\phi^a$ is arbitrary in the interior of $M$, we must have 
\begin{equation}
    E_a[\phi_c]=0
\end{equation}
which is just the \textit{equation of motion}.
To make the spatial boundary contribution, a sufficient condition is that\sidenote{Notice that: \begin{itemize}
    \item This has not been proved to be a necessary condition yet, but the authors of \cite{Harlow:2019yfa} believe it is.
    \item The relation still holds off shell. We can view this as a requirement on the theory, which means that we will only consider theories that satisfies \cref{spatial_boundary_requirement} here and theories that do not satisfy \cref{spatial_boundary_requirement} is not well-defined.
\end{itemize}}
\begin{equation}\label{spatial_boundary_requirement}
    \eval{\left(\Theta+\delta\ell\right)}_\Gamma=\dd{C}
\end{equation}
where $C$ is a local $(d-2)$-form on $\Gamma$ which is locally constructed from terms involving dynamical/background fields.
Although we only defined $C$ on $\Gamma$, we can extend the definition of $C$ to $\Sigma_{\pm}$ in arbitrary way since we will find only values on $\partial\Gamma$ (or $\partial\Sigma_{\pm}$) matter.



\clearpage
\bibliographystyle{jhep}
\bibliography{ref}
\end{document}