\documentclass[10pt]{article}
\usepackage{zsj}

\usepackage{silence}
\WarningFilter{latex}{Marginpar on page}

\allowdisplaybreaks

\hbadness=99999

\newcommand{\me}{\mathrm{e}}
%\newcommand{\rr}{\mathbb{R}}
\newcommand{\ii}{\mathrm{i}}
%\newcommand{\md}{\mathrm{d}}
%\DeclareMathOperator{\im}{im}


\usepackage{marginfix}
\usepackage[export]{adjustbox}

\crefname{claim}{claim}{claims}


\begin{document}
\title{Covariant Phase Space}
\subheader{Notes}
\author{Shangjie Zhou\orcidlink{0000-0001-9576-5011}}
\affiliation{Department of Physics and Astronomy, Purdue University, West Lafayette, IN 47907, USA}
\emailAdd{ZhouShangjie@purdue.edu}
\abstract{\textit{Last updated on: \today}\\Latest version: \url{https://github.com/Zhou-Shangjie/covariant-phase-space/releases}\\Source files: \url{https://github.com/Zhou-Shangjie/covariant-phase-space}\\Personal Website: \url{https://zhou-shangjie.com}}
\maketitle
\phantomsection\addcontentsline{toc}{section}{\protect\numberline{}Introduction}
\section*{Introduction}
This note follows \cite{Harlow:2019yfa}

\section{Hamiltonian Formalism}
The Hamilton equations are
\begin{subequations}
    \begin{align}
        \dot{q}^a & =\pdv{H}{p^a}  \\
        \dot{p}^a & =-\pdv{H}{q^a}
    \end{align}
\end{subequations}
The usual Hamiltonian equation splits time coordinate $t$ from other coordinates which make it hard to maintain covariance.

We view the phase space as an manifold $\mathcal{P}$ equipped with an non-degenrate\sidenote{That $\Omega$ is degenrate means: there is a vector $X\in V(\mathcal{P})$ makes that $\forall Y\in V(\mathcal{P})$, $\Omega(X,Y)=0$.} two-form $\Omega:V(\mathcal{P})\times V(\mathcal{P})\to\mathbb{R}(\mathcal{P})$ called the \textit{symplectic form}.
From $\Omega$, we can induce a map $\Omega_{\text{ind}}:V(\mathcal{P})\to V^\ast(\mathcal{P})$ and for $X,Y\in V(\mathcal{P})$ we define
\begin{equation}
    \Omega_{\text{ind}}(X)(Y)\equiv\Omega(X,Y)
\end{equation}
and notice that $\Omega_{\text{ind}}(X)\in V^\ast$.
We can also define $\Omega^{-1}:V^\ast(\mathcal{P})\times V^\ast(\mathcal{P})\to \mathbb{R}(\mathcal{P})$ via\sidenote{Notice the abuse of notation: $\Omega^{-1}$ here is actually not the inverse of $\Omega$.}
\begin{equation}
    \Omega^{-1}(\omega,\sigma)\equiv\omega(\Omega_{\text{ind}}^{-1}(\sigma))
\end{equation}
where $\omega,\sigma\in V^\ast(\mathcal{P})$.
Given a function $H\in\mathcal{P}$, we can define $X_H\in V(\mathcal{P})$ as
\begin{equation}
    X_H(f)\equiv\Omega^{-1}(\delta f,\delta H)
\end{equation}
where $f\in\mathcal{F}(\mathcal{P})$.

\begin{definition}[Poisson bracket]
    On manifold $\mathcal{P}$, the \textit{Poisson bracket} of two functions  $f,g\in\mathcal{F}(\mathcal{P})$ is defined as
    \begin{equation}
        \pb{f}{g}\equiv\Omega^{-1}(\delta f,\delta g)=\Omega(X_g,X_f).
    \end{equation}
\end{definition}
Now we require that in a dynamic system, the time evolution of a function $f\in\mathcal{F}(\mathcal{P})$ to satisfy
\begin{equation}
    \dot{f}=X_H(f)=\pb{f}{H}.
\end{equation}
The usual Hamiltonian mechanics can be restored if we take
\begin{equation}
    \Omega=\sum_a \delta p^a\wedge \delta q^a
\end{equation}

\section{Construction of Covariant Phase Space}
\begin{definition}[Pre-phase space]
    \textit{Pre-phase space} $\tilde{\mathcal{P}}$ is defined to be the set of solutions of the equations of motion\mn{Boundary conditions are satisfied.}.
\end{definition}
\begin{definition}[Phase space]
    The \textit{phase space} $\mathcal{P}$ is defined as 
    \begin{equation}
        \mathcal{P}\equiv\tilde{\mathcal{P}}/\tilde{G}.
    \end{equation}
\end{definition}
\subsection{Setup}
We consider the following action
\begin{equation}
    S=\int_M L+\int_{\partial M}\ell\label{eq:action}
\end{equation}
If we decompose $\partial M=\Gamma\cup\Sigma_+\cup\sigma_-$, the variation of action around classical solutions should be
\begin{equation}
    \delta S=\int_{\Sigma_+}\Psi-\int_{\sigma_-}\Psi\label{var_S_1}
\end{equation}
\begin{definition}[Configuration space]
    Configuration space $\mathcal{C}$ is defined to be the set of dynamical fields configurations on spacetime obeying boundary conditions at $\Gamma$.
\end{definition}

By integration by parts, any local Lagrangian can be written in the following form
\begin{equation}
    \delta L=E_a\delta \phi^a+\dd{\Theta}
\end{equation}

Therefore, we can write
\begin{equation}
    \delta S=\int_M E_a\delta\phi^a+\int_{\partial M}\left(\delta\ell+\Theta\right).\label{var_S_2}
\end{equation}
For a variation around classical solution $\phi_c$, since \cref{var_S_2} should be in the same form as \cref{var_S_1} and $\delta\phi^a$ is arbitrary in the interior of $M$, we must have
\begin{equation}
    E_a[\phi_c]=0
\end{equation}
which is just the \textit{equation of motion}.
To make the spatial boundary contribution, a sufficient condition is that\sidenote{Notice that: \begin{itemize}
        \item This has not been proved to be a necessary condition yet, but the authors of \cite{Harlow:2019yfa} believe it is.
        \item The relation still holds off shell. We can view this as a requirement on the theory, which means that we will only consider theories that satisfies \cref{spatial_boundary_requirement} here and theories that do not satisfy \cref{spatial_boundary_requirement} is not well-defined.
    \end{itemize}}
\begin{equation}\label{spatial_boundary_requirement}
    \eval{\left(\Theta+\delta\ell\right)}_\Gamma=\dd{C}
\end{equation}
where $C$ is a local $(d-2)$-form on $\Gamma$ which is locally constructed from terms involving dynamical/background fields.
Although we only defined $C$ on $\Gamma$, we can extend the definition of $C$ to $\Sigma_{\pm}$ in arbitrary way since we will find only values on $\partial\Gamma$ (or $\partial\Sigma_{\pm}$) matter.
\begin{definition}[Pre-symplectic current]
    The pre-symplectic current $\omega$ is defined as the pullback of $\delta\Psi$ to $\tilde{\mathcal{P}}$
    \begin{equation}
        \Omega\equiv\eval{\delta\Psi}_{\tilde{\mathcal{P}}}
    \end{equation}
\end{definition}

\begin{definition}[Pre-symplectic form]
    The pre-symplectic form is defined as 
    \begin{equation}
        \tilde{\Omega}\equiv\int_\Sigma \omega
    \end{equation}
    where $\Sigma$ is any Cauchy slice of $M$.
\end{definition}

\subsection{Definition of Covariance}
The variation of a dynamical tensor field under the infinitesimal diffeomorphism generated by a vector field $\xi^\mu$ can be defined as
\begin{equation}
    \delta_{\xi}\phi\equiv\mathcal{L}_\xi \phi
\end{equation}
We also define a vector field on the configuration space
\begin{equation}
    X_\xi\equiv\int\dd[d]{x}\mathcal{L}_\xi \phi^a(x)\frac{\delta}{\delta \phi^a}
\end{equation}
and we have
\begin{equation}
    \delta_\xi \phi^a(x)=\mathcal{L}_{X_\xi}\phi^a(x)=X_{\xi}\cdot\delta\phi^a(x)
\end{equation}
and this variation formula can be generalized to any configuration-space tensor $\sigma$
\begin{equation}
    \delta_\xi\sigma\equiv\mathcal{L}_{X_\xi}\sigma.
\end{equation}

\begin{definition}[Covariance]
    A configuration-space tensor $\sigma$ which is also a spacetime tensor locally constructed out of the dynamical and background fields is \textit{covariant} under the infinitesimal diffeomorphism generated by a vector field $\xi^\mu$ if
    \begin{equation}
        \delta_{\xi}\sigma=\mathcal{L}_\xi \sigma\label{eq:cov_def}
    \end{equation}
    where $\mathcal{L}_\xi$ is a spacetime Lie derivative.
\end{definition}

Covariance of $L$ under the diffeomorphisms generated by a vector field $\xi^\mu$ is not sufficient for those diffeomorphisms to be symmetries.
For a continuous transformation of dynamical fields to be a symmetry, this transformation
\begin{itemize}
    \item must respect the boundary conditions
    \item must preserve the action up to possible boundary terms at $\Sigma_\pm$.
\end{itemize}
The variation of the action \cref{eq:action} by a infinitesimal diffeomorphism under which $L$ is covariant is
\begin{equation}
    \begin{split}
        \delta_\xi S&=\int_M \delta_\xi L+\int_{\partial M}\delta_\xi \ell\\
        &=\int_M \mathcal{L}_\xi L+\int_{\partial M}\delta_\xi \ell\\
        &=\int_M \left(\xi\cdot\dd{L}+\dd(\xi\cdot L)\right)+\int_{\partial M}\delta_\xi \ell\\
        &=\int_{\partial M}\left(\xi\cdot L+\delta_{\xi}\ell\right)
    \end{split}
\end{equation}
where we have used \cref{eq:cov_def} and the fact $\dd{L}=0$ because it is a top form in spacetime.

\begin{remark}
    The covariance of $\ell$ under diffeomorphisms $\xi$ can impose even more restrictions on both $\ell$ and $\xi$.
    See \cite{Harlow:2019yfa} for detailed discussions.
\end{remark}

\subsection{Diffeomorphism charges}

It will be useful to introduce the Noether current\cite{Iyer:1994ys}
\begin{definition}[Noether current]
    The Noether current is defined as
    \begin{equation}
        J_\xi\equiv X_\xi\cdot\Theta-\xi\cdot L
    \end{equation}
    and $J_\xi$ is a $(d-1,0)$-form.
\end{definition}

\begin{claim}
    If $L$ is covariant under $\xi$ then $J_{\xi}$ is closed as a spacetime form, which means $\dd{J_\xi}=0$.
\end{claim}
\begin{proof}
    \begin{equation}
        \begin{split}
            \dd{J_\xi}=\dd{X_\xi\cdot\Theta}-\dd{\xi\cdot L}
        \end{split}
    \end{equation}
\end{proof}


\begin{claim}
    $-X_\xi\cdot\tilde{\Omega}$ is exact with respect to $\delta$, which means
    \begin{equation}
        -X_\xi\cdot\tilde{\Omega}=\delta H_{\xi}
    \end{equation}
    where $H_{\xi}$ is some function on $\tilde{\mathcal{P}}$.
\end{claim}
\begin{proof}
    \begin{equation}
        \begin{split}
            -X_\xi\cdot\tilde{\Omega}=
        \end{split}
    \end{equation}
    We can see that
    \begin{equation}
        -X_\xi\cdot\tilde{\Omega}=\delta H_\xi
    \end{equation}
    where
    \begin{equation}
        H_\xi\equiv\int_{\Sigma}J_{\xi}+\int_{\partial\Sigma}(\xi\cdot\ell-X_\xi\cdot C)+\text{const.}
    \end{equation}

\end{proof}

\begin{claim}
    $H_\xi$ is independent of the choice of Cauchy slice $\Sigma$.
\end{claim}
\begin{proof}
    1
\end{proof}

\section{Examples}
\begin{example}[Partical mechanics]
    1
\end{example}

\begin{example}[Two-derivative scalar field]
    1
\end{example}

\begin{example}[Maxwell theory]
    1
\end{example}

\begin{example}[General relativity]
    1
\end{example}

\clearpage
\bibliographystyle{jhep}
\bibliography{ref}
\end{document}